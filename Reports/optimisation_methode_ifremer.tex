\documentclass[]{article}
\usepackage{lmodern}
\usepackage{amssymb,amsmath}
\usepackage{ifxetex,ifluatex}
\usepackage{fixltx2e} % provides \textsubscript
\ifnum 0\ifxetex 1\fi\ifluatex 1\fi=0 % if pdftex
  \usepackage[T1]{fontenc}
  \usepackage[utf8]{inputenc}
\else % if luatex or xelatex
  \ifxetex
    \usepackage{mathspec}
  \else
    \usepackage{fontspec}
  \fi
  \defaultfontfeatures{Ligatures=TeX,Scale=MatchLowercase}
\fi
% use upquote if available, for straight quotes in verbatim environments
\IfFileExists{upquote.sty}{\usepackage{upquote}}{}
% use microtype if available
\IfFileExists{microtype.sty}{%
\usepackage{microtype}
\UseMicrotypeSet[protrusion]{basicmath} % disable protrusion for tt fonts
}{}
\usepackage[margin=1in]{geometry}
\usepackage{hyperref}
\hypersetup{unicode=true,
            pdftitle={Optimisation vers les méthodes de dosages proposées par l'IFREMER},
            pdfauthor={Antoine Batigny \& Engels Guyliann},
            pdfborder={0 0 0},
            breaklinks=true}
\urlstyle{same}  % don't use monospace font for urls
\usepackage{color}
\usepackage{fancyvrb}
\newcommand{\VerbBar}{|}
\newcommand{\VERB}{\Verb[commandchars=\\\{\}]}
\DefineVerbatimEnvironment{Highlighting}{Verbatim}{commandchars=\\\{\}}
% Add ',fontsize=\small' for more characters per line
\usepackage{framed}
\definecolor{shadecolor}{RGB}{248,248,248}
\newenvironment{Shaded}{\begin{snugshade}}{\end{snugshade}}
\newcommand{\KeywordTok}[1]{\textcolor[rgb]{0.13,0.29,0.53}{\textbf{#1}}}
\newcommand{\DataTypeTok}[1]{\textcolor[rgb]{0.13,0.29,0.53}{#1}}
\newcommand{\DecValTok}[1]{\textcolor[rgb]{0.00,0.00,0.81}{#1}}
\newcommand{\BaseNTok}[1]{\textcolor[rgb]{0.00,0.00,0.81}{#1}}
\newcommand{\FloatTok}[1]{\textcolor[rgb]{0.00,0.00,0.81}{#1}}
\newcommand{\ConstantTok}[1]{\textcolor[rgb]{0.00,0.00,0.00}{#1}}
\newcommand{\CharTok}[1]{\textcolor[rgb]{0.31,0.60,0.02}{#1}}
\newcommand{\SpecialCharTok}[1]{\textcolor[rgb]{0.00,0.00,0.00}{#1}}
\newcommand{\StringTok}[1]{\textcolor[rgb]{0.31,0.60,0.02}{#1}}
\newcommand{\VerbatimStringTok}[1]{\textcolor[rgb]{0.31,0.60,0.02}{#1}}
\newcommand{\SpecialStringTok}[1]{\textcolor[rgb]{0.31,0.60,0.02}{#1}}
\newcommand{\ImportTok}[1]{#1}
\newcommand{\CommentTok}[1]{\textcolor[rgb]{0.56,0.35,0.01}{\textit{#1}}}
\newcommand{\DocumentationTok}[1]{\textcolor[rgb]{0.56,0.35,0.01}{\textbf{\textit{#1}}}}
\newcommand{\AnnotationTok}[1]{\textcolor[rgb]{0.56,0.35,0.01}{\textbf{\textit{#1}}}}
\newcommand{\CommentVarTok}[1]{\textcolor[rgb]{0.56,0.35,0.01}{\textbf{\textit{#1}}}}
\newcommand{\OtherTok}[1]{\textcolor[rgb]{0.56,0.35,0.01}{#1}}
\newcommand{\FunctionTok}[1]{\textcolor[rgb]{0.00,0.00,0.00}{#1}}
\newcommand{\VariableTok}[1]{\textcolor[rgb]{0.00,0.00,0.00}{#1}}
\newcommand{\ControlFlowTok}[1]{\textcolor[rgb]{0.13,0.29,0.53}{\textbf{#1}}}
\newcommand{\OperatorTok}[1]{\textcolor[rgb]{0.81,0.36,0.00}{\textbf{#1}}}
\newcommand{\BuiltInTok}[1]{#1}
\newcommand{\ExtensionTok}[1]{#1}
\newcommand{\PreprocessorTok}[1]{\textcolor[rgb]{0.56,0.35,0.01}{\textit{#1}}}
\newcommand{\AttributeTok}[1]{\textcolor[rgb]{0.77,0.63,0.00}{#1}}
\newcommand{\RegionMarkerTok}[1]{#1}
\newcommand{\InformationTok}[1]{\textcolor[rgb]{0.56,0.35,0.01}{\textbf{\textit{#1}}}}
\newcommand{\WarningTok}[1]{\textcolor[rgb]{0.56,0.35,0.01}{\textbf{\textit{#1}}}}
\newcommand{\AlertTok}[1]{\textcolor[rgb]{0.94,0.16,0.16}{#1}}
\newcommand{\ErrorTok}[1]{\textcolor[rgb]{0.64,0.00,0.00}{\textbf{#1}}}
\newcommand{\NormalTok}[1]{#1}
\usepackage{graphicx,grffile}
\makeatletter
\def\maxwidth{\ifdim\Gin@nat@width>\linewidth\linewidth\else\Gin@nat@width\fi}
\def\maxheight{\ifdim\Gin@nat@height>\textheight\textheight\else\Gin@nat@height\fi}
\makeatother
% Scale images if necessary, so that they will not overflow the page
% margins by default, and it is still possible to overwrite the defaults
% using explicit options in \includegraphics[width, height, ...]{}
\setkeys{Gin}{width=\maxwidth,height=\maxheight,keepaspectratio}
\IfFileExists{parskip.sty}{%
\usepackage{parskip}
}{% else
\setlength{\parindent}{0pt}
\setlength{\parskip}{6pt plus 2pt minus 1pt}
}
\setlength{\emergencystretch}{3em}  % prevent overfull lines
\providecommand{\tightlist}{%
  \setlength{\itemsep}{0pt}\setlength{\parskip}{0pt}}
\setcounter{secnumdepth}{0}
% Redefines (sub)paragraphs to behave more like sections
\ifx\paragraph\undefined\else
\let\oldparagraph\paragraph
\renewcommand{\paragraph}[1]{\oldparagraph{#1}\mbox{}}
\fi
\ifx\subparagraph\undefined\else
\let\oldsubparagraph\subparagraph
\renewcommand{\subparagraph}[1]{\oldsubparagraph{#1}\mbox{}}
\fi

%%% Use protect on footnotes to avoid problems with footnotes in titles
\let\rmarkdownfootnote\footnote%
\def\footnote{\protect\rmarkdownfootnote}

%%% Change title format to be more compact
\usepackage{titling}

% Create subtitle command for use in maketitle
\newcommand{\subtitle}[1]{
  \posttitle{
    \begin{center}\large#1\end{center}
    }
}

\setlength{\droptitle}{-2em}

  \title{Optimisation vers les méthodes de dosages proposées par l'IFREMER}
    \pretitle{\vspace{\droptitle}\centering\huge}
  \posttitle{\par}
    \author{Antoine Batigny \& Engels Guyliann}
    \preauthor{\centering\large\emph}
  \postauthor{\par}
    \date{}
    \predate{}\postdate{}
  

\begin{document}
\maketitle

\begin{Shaded}
\begin{Highlighting}[]
\NormalTok{SciViews}\OperatorTok{::}\NormalTok{R}
\end{Highlighting}
\end{Shaded}

\begin{verbatim}
## -- Attaching packages ------------------------------------------------------------------------------- SciViews::R 1.0.0 --
\end{verbatim}

\begin{verbatim}
## √ SciViews  1.0.0      √ readr     1.1.1 
## √ svMisc    1.1.0      √ tidyr     0.8.1 
## √ forcats   0.3.0      √ tibble    1.4.2 
## √ stringr   1.3.1      √ ggplot2   3.0.0 
## √ dplyr     0.7.6      √ tidyverse 1.2.1 
## √ purrr     0.2.5      √ MASS      7.3.50
\end{verbatim}

\begin{verbatim}
## -- Conflicts ------------------------------------------------------------------------------------ tidyverse_conflicts() --
## x dplyr::filter() masks stats::filter()
## x dplyr::lag()    masks stats::lag()
## x dplyr::select() masks MASS::select()
\end{verbatim}

\begin{Shaded}
\begin{Highlighting}[]
\KeywordTok{library}\NormalTok{(chart)}
\end{Highlighting}
\end{Shaded}

\begin{verbatim}
## Loading required package: lattice
\end{verbatim}

\begin{Shaded}
\begin{Highlighting}[]
\KeywordTok{library}\NormalTok{(flow)}
\KeywordTok{library}\NormalTok{(econum)}
\end{Highlighting}
\end{Shaded}

\section{L'eau de lavage
inter-échantillon}\label{leau-de-lavage-inter-echantillon}

Précédemment, les méthodes de dosage utilisées au laboratoire d'ECONUM
employaient une eau SSW de lavage inter-échantillon. La méthode IFREMER
utilise par contre de l'eau osmosée ultra-pure avec un échantillon en
eau salée pour appliquer un facteur de correction par rapport à l'eau
osmosée.

\begin{Shaded}
\begin{Highlighting}[]
\KeywordTok{repos_load}\NormalTok{(}\StringTok{"../Data/calibration/aa3/180711B-inorga_2018-07-11_13.29.26_5B454880_aa3.RData"}\NormalTok{)}
\NormalTok{nutri1 <-}\StringTok{ }\NormalTok{EcoNumData_aa3}
\NormalTok{nutri1 <-}\StringTok{ }\KeywordTok{filter}\NormalTok{(nutri1, sample_type }\OperatorTok{==}\StringTok{ "SAMP"}\NormalTok{)}
\NormalTok{nutri1}\OperatorTok{$}\NormalTok{lavage <-}\StringTok{ "ssw"}

\KeywordTok{repos_load}\NormalTok{(}\StringTok{"../Data/calibration/aa3/180711C-inorga_2018-07-11_15.02.30_5B454880_aa3.RData"}\NormalTok{)}
\NormalTok{nutri2 <-}\StringTok{ }\NormalTok{EcoNumData_aa3}
\NormalTok{nutri2 <-}\StringTok{ }\KeywordTok{filter}\NormalTok{(nutri2, sample_type }\OperatorTok{==}\StringTok{ "SAMP"}\NormalTok{)}
\NormalTok{nutri2}\OperatorTok{$}\NormalTok{lavage <-}\StringTok{ "milliq"}

\NormalTok{nutri <-}\StringTok{ }\KeywordTok{bind_rows}\NormalTok{(nutri1, nutri2)}
\KeywordTok{rm}\NormalTok{(nutri1, nutri2)}


\KeywordTok{repos_load}\NormalTok{(}\StringTok{"../Data/calibration/aa3/180716A-inorga_2018-07-16_11.44.37_5B4BE000_aa3.RData"}\NormalTok{)}
\NormalTok{nutri3 <-}\StringTok{ }\NormalTok{EcoNumData_aa3}
\NormalTok{nutri3 <-}\StringTok{ }\KeywordTok{filter}\NormalTok{(nutri3, sample_type }\OperatorTok{==}\StringTok{ "SAMP"}\NormalTok{)}
\NormalTok{nutri3}\OperatorTok{$}\NormalTok{lavage <-}\StringTok{ "milliq"}

\KeywordTok{repos_load}\NormalTok{(}\StringTok{"../Data/calibration/aa3/180716B-inorga_2018-07-16_13.53.06_5B4BE000_aa3.RData"}\NormalTok{)}
\NormalTok{nutri4 <-}\StringTok{ }\NormalTok{EcoNumData_aa3}
\NormalTok{nutri4 <-}\StringTok{ }\KeywordTok{filter}\NormalTok{(nutri4, sample_type }\OperatorTok{==}\StringTok{ "SAMP"}\NormalTok{)}
\NormalTok{nutri4}\OperatorTok{$}\NormalTok{lavage <-}\StringTok{ "ssw"}

\NormalTok{nutri2 <-}\StringTok{ }\KeywordTok{bind_rows}\NormalTok{(nutri3, nutri4)}
\end{Highlighting}
\end{Shaded}

\begin{verbatim}
## Warning in bind_rows_(x, .id): Unequal factor levels: coercing to character
\end{verbatim}

\begin{verbatim}
## Warning in bind_rows_(x, .id): binding character and factor vector,
## coercing into character vector

## Warning in bind_rows_(x, .id): binding character and factor vector,
## coercing into character vector
\end{verbatim}

\begin{Shaded}
\begin{Highlighting}[]
\KeywordTok{rm}\NormalTok{(EcoNumData_aa3, nutri3, nutri4)}
\end{Highlighting}
\end{Shaded}

\subsection{PO4}\label{po4}

Le dosage des mêmes échantillons avec une eau de lavage
inter-échantillon milliQ ou SSW indique des concentrations similaires.

\begin{Shaded}
\begin{Highlighting}[]
\KeywordTok{chart}\NormalTok{(nutri2, PO4_conc }\OperatorTok{~}\StringTok{ }\NormalTok{sample }\OperatorTok\StringTok{ }\NormalTok{lavage ) }\OperatorTok{+}
\StringTok{  }\KeywordTok{geom_point}\NormalTok{(}\DataTypeTok{position=}\KeywordTok{position_dodge}\NormalTok{(}\DataTypeTok{width =} \FloatTok{0.9}\NormalTok{)) }\OperatorTok{+}
\StringTok{  }\KeywordTok{labs}\NormalTok{(}\DataTypeTok{x =} \StringTok{"Echantillon"}\NormalTok{, }\DataTypeTok{y =} \StringTok{"Ions phosphate [µmol/L]"}\NormalTok{, }\DataTypeTok{color =} \StringTok{"Eau de }\CharTok{\textbackslash{}n}\StringTok{ lavage"}\NormalTok{) }
\end{Highlighting}
\end{Shaded}

\includegraphics{optimisation_methode_ifremer_files/figure-latex/unnamed-chunk-2-1.pdf}

\subsection{NO3}\label{no3}

Le dosage des mêmes échantillons avec une eau de lavage
inter-échantillon milliQ ou SSW indique des concentrations différentes.
Il semble que l'eau milliQ soit plus instable que la SSW.

\begin{Shaded}
\begin{Highlighting}[]
\KeywordTok{chart}\NormalTok{(nutri2, NO3_conc }\OperatorTok{~}\StringTok{ }\NormalTok{sample }\OperatorTok{|}\NormalTok{lavage ) }\OperatorTok{+}
\StringTok{  }\KeywordTok{geom_point}\NormalTok{()}
\end{Highlighting}
\end{Shaded}

\includegraphics{optimisation_methode_ifremer_files/figure-latex/unnamed-chunk-3-1.pdf}

\begin{Shaded}
\begin{Highlighting}[]
\KeywordTok{chart}\NormalTok{(nutri2, NO3_conc }\OperatorTok{~}\StringTok{ }\NormalTok{sample }\OperatorTok\StringTok{ }\NormalTok{lavage ) }\OperatorTok{+}
\StringTok{  }\KeywordTok{geom_point}\NormalTok{(}\DataTypeTok{position=}\KeywordTok{position_dodge}\NormalTok{(}\DataTypeTok{width =} \FloatTok{0.9}\NormalTok{)) }\OperatorTok{+}
\StringTok{  }\KeywordTok{labs}\NormalTok{(}\DataTypeTok{x =} \StringTok{"Echantillon"}\NormalTok{, }\DataTypeTok{y =} \StringTok{"Ions nitrate & nitrite [µmol/L]"}\NormalTok{, }\DataTypeTok{color =} \StringTok{"Eau de }\CharTok{\textbackslash{}n}\StringTok{ lavage"}\NormalTok{) }
\end{Highlighting}
\end{Shaded}

\includegraphics{optimisation_methode_ifremer_files/figure-latex/unnamed-chunk-3-2.pdf}

\subsection{NH4}\label{nh4}

Le dosage des mêmes échantillons avec une eau de lavage
inter-échantillon milliQ ou SSW indique des concentrations très
différentes. Il semble que l'eau milliQ contiennent un teneur en NH4 non
négligeable.

\begin{Shaded}
\begin{Highlighting}[]
\KeywordTok{chart}\NormalTok{(nutri2, NH4_conc }\OperatorTok{~}\StringTok{ }\NormalTok{sample }\OperatorTok\StringTok{ }\NormalTok{lavage ) }\OperatorTok{+}
\StringTok{  }\KeywordTok{geom_point}\NormalTok{(}\DataTypeTok{position=}\KeywordTok{position_dodge}\NormalTok{(}\DataTypeTok{width =} \FloatTok{0.9}\NormalTok{)) }\OperatorTok{+}
\StringTok{  }\KeywordTok{labs}\NormalTok{(}\DataTypeTok{x =} \StringTok{"Echantillon"}\NormalTok{, }\DataTypeTok{y =} \StringTok{"Ions ammonium [µmol/L]"}\NormalTok{, }\DataTypeTok{color =} \StringTok{"Eau de }\CharTok{\textbackslash{}n}\StringTok{ lavage"}\NormalTok{) }
\end{Highlighting}
\end{Shaded}

\includegraphics{optimisation_methode_ifremer_files/figure-latex/unnamed-chunk-4-1.pdf}

\section{Conclusion et perspectives}\label{conclusion-et-perspectives}

Les concentrations en Azote (natrite, nitrite, ammonium) semblent plus
sensible à la modification


\end{document}
